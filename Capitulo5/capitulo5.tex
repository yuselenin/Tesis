\chapter{ADMINISTRACIÓN DEL PROYECTO DE INVESTIGACIÓN }
\section{Cronograma del proyecto}

%Diseñar e implementar la heurística de construcción El vecino más cercano y la heurística demejoría 2-opt y/o otros, utilizado el lenguaje de programación Python.
%    Plantear valores de entrada y salida
%    Implementar el modelo
%    Diseñar diagramas de flujo
%    Codificar la heurística de construcción
%    Codificar la heurística de mejoría
%Implementar la Metaheurística Búsqueda Tabú para mejorar la solución de la heurística de mejoría, utilizado Python.
%    Codificar
%Validar el algoritmo con técnicas estadísticas utilizando instancias artificiales encontradas en laliteratura y datos reales de la región Puno.
%    Validar con técnicas estadísticas utilizando instancias artificiales encontradas en la literatura
%        * Posiblemente no se pueda por que el algoritmo es mas detallado
%    Validar el algoritmo utilizando datos reales de la región Puno
%        Recolectará los datos
%        Si faltan datos se pretende rellenar de forma aletoria(horario de apertura)
%Desarrollar una aplicación web que permita al turista ingresar sus preferencias y visualizar el itinerario detallado utilizando el framework web Django y Api de Google Maps
%    *Recursivo
%    Requiriemientos
%    ~Prototipo
%    Codificacion
%    Pruebas(Volver a mejorar el modelo)
%    Despliegue
%    ~Casos de uso
%    +Agil
%Documentación(Actividad trasversal)

%Para la metodologia se hara un diagrama de bloues(en la que se pueda indicar que se puedan regreasr)

\begin{table}[ht]
\caption{Cronograma de actividades para la realización de la investigación propuesta}
\label{tab:cronograma}
\newcolumntype{L}[1]{>{\raggedright\arraybackslash}p{#1}}
\begin{tabular}{L{1cm}L{8cm}L{1cm}L{2cm}L{2cm}}
\hline
\textbf{Código}&\textbf{Nombre}&\textbf{Días}&\textbf{Fecha inicio}&\textbf{Fecha fin}\\
\hline
\textbf{1.} & \textbf{\thesistitle} & \textbf{35} & \textbf{7/12/2018} & \textbf{28/01/2019}\\
\textbf{1.1.} & \textbf{Objetivo 1} & \textbf{7} & \textbf{7/12/2018} & \textbf{17/12/2018}\\
1.1.1. & Codificar el algoritmo del Vecino más Cercano & 7 & 7/12/2018 & 17/12/2018\\
1.1.2. & Codificar el algoritmo de 2-opt & 7 & 7/12/2018 & 17/12/2018\\

\textbf{1.2.} & \textbf{Objetivo 2} & \textbf{5} & \textbf{18/12/2018} & \textbf{24/12/2018}\\
1.2.1. & Codificar el algoritmo de Búsqueda Tabú & 5 & 18/12/2018 & 24/12/2018\\

\textbf{1.3.} & \textbf{Objetivo 3} & \textbf{4} & \textbf{25/12/2018} & \textbf{31/12/2018}\\
1.3.1. & Recolectar datos reales de la región Puno& 4 & 25/12/2018 & 31/12/2018\\
1.3.2. & Validar el algoritmo utilizando datos reales & 4 & 25/12/2018 & 31/12/2018\\
1.3.3. & Recolectar datos artificiales & 4 & 25/12/2018 & 31/12/2018\\
1.3.4. & Validar el algoritmo utilizando instancias artificiales & 4 & 25/12/2018 & 31/12/2018\\

\textbf{1.4.} & \textbf{Objetivo 4} & \textbf{19} & \textbf{01/01/2019} & \textbf{28/01/2019}\\
1.4.1. & Planeación & 19 & 01/01/2019 & 28/01/2019\\
1.4.2. & Diseño & 19 & 01/01/2019 & 28/01/2019\\
1.4.3. & Codificación & 19 & 01/01/2019 & 28/01/2019\\
1.4.4. & Pruebas & 19 & 01/01/2019 & 28/01/2019\\
\hline 
\end{tabular}
\end{table}
\section{Presupuesto}
En la tabla \ref{tab:presupuesto} se muestra a continuación se presenta el presupuesto detallado para la realización de la investigación propuesta.

\begin{table}[ht]
\caption{Detalle de presupuesto para la realización de la investigación propuesta}
\label{tab:presupuesto}
\begin{tabular}{lclrr}
\hline
\textbf{Descripción}&\textbf{Cantidad}&\textbf{U. medida}&\textbf{Costo unitario S/.}&\textbf{Costo total S/.}\\
\hline
\textbf{Personal} & & & & \textbf{3000.00}\\
Analista & 1 & Mes & 1000.00 & 1000.00\\
Programador & 2 & Mes & 1000.00 & 2000.00\\

\textbf{Materiales y equipos} & & & & \textbf{2120.00}\\
Laptop & 1 & Unidad & 2000.00 & 2000.00\\
Anillados de copias & 12 & Unidad & 10.00 & 120.00\\

\textbf{Servicios} & & & & \textbf{160.00}\\
Servicio de internet & 2 & Mes & 30.00 & 60.00\\
Viáticos y movilidad & 2 & Mes & 50.00 & 100.00\\
Biblioteca & 2 & Mes & 0.00 & 0.00\\
\textbf{Imprevistos} & & & & \textbf{200.00}\\
\textbf{Total} & & & & \textbf{5480.00}\\
\hline 
\end{tabular}
\end{table}

El coste del proyecto es de 5480.00 nuevos soles con 0/100 centavos, de las cuales será financiamiento por el investigador de forma monetario y no monetario.