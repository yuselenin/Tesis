\usepackage[margin=1in]{geometry} % indica todo los margenes de 1 pulgada(2.54cm)
\usepackage[T1]{fontenc} % revisar uso...
\usepackage[utf8x]{inputenc} % para el soporte de las tildes,etc...;el oficial es "utf8" pero no soporta "°"
\usepackage[spanish,es-tabla]{babel} % revisar uso...
\usepackage{amsmath} % revisar uso...
\usepackage{amssymb,amsfonts,latexsym,cancel} % revisar uso...
\usepackage{array} % revisar uso...
\usepackage{bm} % revisar uso...
\usepackage{float} % revisar uso...
\usepackage{fancyhdr} % revisar uso...
\usepackage{graphicx} % revisar uso...
\usepackage{epstopdf}% revisar uso...
\usepackage[colorlinks = true,
            linkcolor = black,
            citecolor = black,
            urlcolor = blue]{hyperref} % modifica los estilos de las urls
\setcounter{MaxMatrixCols}{40}  % revisar uso...
\usepackage{multicol} % revisar uso...
\usepackage{subfigure} % revisar uso...
\usepackage{titling} % revisar uso...
\newcolumntype{E}{>{$}c<{$}} % revisar uso...
\usepackage{apacite} % usado para citar y el formato de la lista de referencias
\raggedright % alinea a la izquierda
\usepackage{pdfpages} % para insertar pdfs(anexos)
\renewcommand{\thechapter}{\Roman{chapter}} % cambia la numeración de los capítulos en números romanos
\usepackage{pdflscape} % para cambiar una pagina en horizontal
\usepackage{longtable} % para acomodar en varias paginas una tabla grande
% Para el uso de esquemas https://tex.stackexchange.com/questions/112175/brace-diagram-in- % no entiendo...

\usepackage{titlesec} % revisar uso...(creo que title format lo usa)
\titlespacing{\chapter}{0pt}{-50pt}{*0} % quita espacio antes del capitulo
\titlespacing{\section}{0pt}{*0}{*0} % quita espacio antes y despues
\titlespacing{\subsection}{0pt}{*0}{*0} % quita espacio antes y despues
\titlespacing{\subsubsection}{0pt}{*0}{3pt} % quita espacio antes y despues
\titlespacing{\paragraph}{0pt}{*0}{3pt} % quita espacio antes y despues
\titlespacing{\subparagraph}{0pt}{*0}{2pt} % quita espacio antes y despues
% Son necesario las dos secciones siguiente..
\titleformat{\chapter}[display] % oculta el texto de "Capitulo..."
    {\normalfont\centering} % centrado
    {}
    {0pt}
    {\thechapter.\quad} % doc para sepadores https://tex.stackexchange.com/questions/74353/what-commands-are-there-for-horizontal-spacing?noredirect=1&lq=1

\titleformat{name=\chapter,numberless}[display] % revisar uso...
    {\normalfont\centering\bfseries}
    {}
    {0pt}
    {}
\titleformat{\section}[block]
    {\normalfont\centering\bfseries}
    {\thesection.}
    {1em}
    {} % estilos para la sección(nivel 1)
\titleformat{\subsection}[block]
    {\normalfont\bfseries}
    {\thesubsection.}
    {1em}
    {} % estilos para subsección(nivel 2)
% estilos para subsubsección(nivel 3)
\titleformat{\subsubsection}[runin]% runin puts it in the same paragraph
    {\normalfont\bfseries}% formatting commands to apply to the whole heading
    {\thesubsubsection.}% the label and number
    {}% space between label/number and subsection title
    {\hspace{0.5in}}% formatting commands applied just to subsection title
    [.]% punctuation or other commands following subsection title
%\newcommand{\sectionbreak}{\clearpage}
%\renewcommand{\thechapter}{} % para quitar los numeros romanos
\renewcommand\thesection{\@arabic\c@section} % pone la numeracion sin el del capitulo según https://stackoverflow.com/questions/3978203/how-do-i-keep-my-section-numbering-in-latex-but-just-hide-it
\usepackage{setspace}
\setlength{\parindent}{0.5in} % pone sangria de 0.5 pulgada(1.27cm)
\titleformat{\paragraph}[runin]
    {\normalfont\bfseries\itshape}
    {}
    {}
    {\hspace{0.5in}}
    [.] % estilos para párrafo(nivel 4)
\titleformat{\subparagraph}[runin]
    {\normalfont\itshape}
    {}
    {}
    {\hspace{0.5in}}
    [.] % estilos para (nivel 5)
\usepackage{mathptmx}

% Sección que se encarga de modificar los estilos de la descripcion de las figuras y tablas.
% extraido de https://tex.stackexchange.com/questions/254254/how-to-set-figures-and-tables-captions-in-apa-style-without-using-apa6-class
\usepackage{caption}
\DeclareCaptionLabelSeparator*{spaced}{\\[1ex]}
\captionsetup[table]{textfont=it,format=plain,justification=justified,
  singlelinecheck=false,labelsep=spaced,skip=.3em}
\captionsetup[figure]{labelsep=period,labelfont=it,justification=justified,
  singlelinecheck=false,font=doublespacing}
% pone la numeracion de acuerdo a todo el doumentos no por capitulos
%\renewcommand{\thefigure}{\thesection-\arabic{figure}}
%\renewcommand{\thefigure}{\arabic{figure}}
%\renewcommand{\thetable}{\arabic{table}}
\usepackage{chngcntr}
\counterwithout{figure}{chapter}
\counterwithout{table}{chapter}
% TODO: Falta poner "Tabla..." en el indice, tambien de la imagenes
% Tabla 1. El título debe ser breve y descriptivo.	3

\let\oldquote\quote % es necesario para que funciona la siguiente indicación
\renewcommand\quote{\setlength{\leftmargini}{.5in}\oldquote} % cambia la identacion de los bloques "quote"

\usepackage{multirow,tabularx} % usado para poner multiples filas y columnas https://tex.stackexchange.com/questions/78169/table-rowspan-and-colspan

\usepackage{tocbasic}

\DeclareTOCStyleEntry[
  entrynumberformat=\entrynumberwithprefix{\figurename},
  dynnumwidth,
  numsep=0.5em
]{tocline}{figure}
\DeclareTOCStyleEntry[
  entrynumberformat=\entrynumberwithprefix{\tablename},
  dynnumwidth,
  numsep=0.5em
]{tocline}{table}
\newcommand\entrynumberwithprefix[2]{#1\hfill#2.\hfill}

% para quitar espacios de las listas, https://texfaq.org/FAQ-complist
\usepackage{enumitem}
